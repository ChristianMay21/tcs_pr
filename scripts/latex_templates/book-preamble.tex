\hbadness=99999
\def\tuftefigures{1}
\def\usehypertarget{0}



%%%%%%%%%%%%%%%%%%%%%%%%%%%%%%%%%%%%%%%%%%%%%%%%%%%%%%%%%%%%%%%%%%
% Load packages


\usepackage{shorttoc}
\renewcommand*\contentsname{Contents (detailed)}

\extrafloats{100}

\usepackage{marginfix}  % Boaz: try to fix margin notes overlap
\usepackage{etoolbox,xpatch}
\usepackage{ifxetex,ifluatex}

%%% standard math packages
\usepackage{amsmath,amssymb,mathtools,mathrsfs}

%% ams theorem configuration
\usepackage{amsthm}

% include pages from pdf
\usepackage{pdfpages}


\usepackage{fontawesome}


\usepackage[
    type={CC},
    modifier={by-nc-nd},
    version={4.0},
]{doclicense}

\usepackage{datetime}

\usepackage{longtable,booktabs} % {tabu}

\usepackage{units}

% Typesets the font size, leading, and measure in the form of 10/12x26 pc.
%\newcommand{\measure}[3]{#1/#2$\times$\unit[#3]{pc}}
% Boaz: commented out

%%
% Prints a trailing space in a smart way.
\usepackage{xspace}

\usepackage{tocloft} % for table of contents


\usepackage{textcase}
\def\ucase{\MakeTextUppercase}

%%%%%%%%%%%%%%%%%%%%%%%%%%%%%%%%%%%%%%%%
% Environment for code listing
\usepackage{minted}

\newminted[code]{nand}{breaklines,mathescape}
\newminted[framedcode]{nand}{frame=single,breaklines,mathescape}

\newmintinline[codeinline]{nand}{breaklines,mathescape}

\makeatletter
\xpatchcmd{\codeinline}{\minted@fvset}{\minted@fvset\dontdofcolorbox}{}{}
\makeatother



\makeatletter
\newlength{\fullwidthlength}
\AtBeginDocument{\setlength{\fullwidthlength}{\@tufte@fullwidth}}
\makeatother

%\renewenvironment{fullwidth}{\begin{minipage}{\fullwidthlength}}{\end{minipage}}

\makeatletter
\AtBeginEnvironment{minted}{\dontdofcolorbox}
\def\dontdofcolorbox{\renewcommand\fcolorbox[4][]{##4}}
\makeatother


%%%%%%%%%%%%%%%%%%%%%%%%%%%%%%%%%%%%%%%%%%
% Graphics
\usepackage{graphicx,grffile}
\makeatletter
\def\maxwidth{\ifdim\Gin@nat@width>\linewidth\linewidth\else\Gin@nat@width\fi}
\def\maxheight{\ifdim\Gin@nat@height>\textheight\textheight\else\Gin@nat@height\fi}
\makeatother
% Scale images if necessary, so that they will not overflow the page
% margins by default, and it is still possible to overwrite the defaults
% using explicit options in \includegraphics[width, height, ...]{}
\setkeys{Gin}{width=\maxwidth,height=\maxheight,keepaspectratio}



%%%%%%%%%%%%%%%%%%%%%%%%%%%%%%%%%%%%%%%%%%%
% Macro to put caption below figure
% Taken from
% https://tex.stackexchange.com/questions/229308/combining-tufte-latex-and-threeparttable/229419#229419



\usepackage{etoolbox}

\makeatletter
\newif\if@tufte@margtab\@tufte@margtabfalse
\AtBeginEnvironment{margintable}{\@tufte@margtabtrue}
\AtEndEnvironment{margintable}{\@tufte@margtabfalse}
\newcommand{\classiccaptionstyle}{%
    \long\def\@caption##1[##2]##3{%
        \par
        \addcontentsline{\csname ext@##1\endcsname}{##1}%
        {\protect\numberline{\csname the##1\endcsname}{\ignorespaces ##2}}%
        \begingroup
        \@parboxrestore
        \if@minipage
        \@setminipage
        \fi
        \normalsize
        \@makecaption{\csname fnum@##1\endcsname}{\ignorespaces ##3}\par
        \endgroup}
    \long\def\@makecaption##1##2{%
        \vskip\abovecaptionskip
        \sbox\@tempboxa{\@tufte@caption@font##1: ##2}%
        \ifdim \wd\@tempboxa >\hsize
        \@tufte@caption@font\if@tufte@margtab\@tufte@caption@justification\fi##1: ##2\par
        \else
        \global \@minipagefalse
        \hb@xt@\hsize{\hfil\box\@tempboxa\hfil}%
        \fi
        \vskip\belowcaptionskip}
    %   \setcaptionfont{\normalfont}
    \let\caption\@tufte@orig@caption%
    \let\label\@tufte@orig@label}
\makeatother


%%%%%%%%%%%%%%%%%%%%%%%%%%%%%%%%%%%%%%%%%%%%

\setsidenotefont{\footnotesize}
\setcaptionfont{\footnotesize}
\setmarginnotefont{\footnotesize}
\setcitationfont{\footnotesize}



%%%%%%%%%%%%%%%%%%%%%%%%%%%%%%%%%%%%%%%%%%%%

\ifnum\tuftefigures=1

\else

% Reset figure and float environments to their original
% (non-Tufte) styles.
\makeatletter
\renewenvironment{figure}[1][htbp]{%
  \@tufte@orig@float{figure}[#1]%
}{%
  \@tufte@orig@endfloat
}

\renewenvironment{table}[1][htbp]{%
  \@tufte@orig@float{table}[#1]%
}{%
  \@tufte@orig@endfloat
}
\makeatother

\usepackage[font={sf}]{caption}

\fi



\makeatletter
\renewcommand{\fnum@figure}{{\bf\sffamily Figure \thefigure}}

\renewcommand{\fnum@table}{{\bf\sffamily Table \thetable}}

\makeatother





\ifx\pdfsuppressptexinfo\undefined
\else
\pdfsuppressptexinfo=-1
\pdftrailerid{}
\pdfinfoomitdate=1
\fi





%%%%%%%%%%%%%%%%%%%%%%%%%%%%%%%%%%%%%%%%%%%%%%%%%%%%%%%%%%%
% Hyperref setup

\RequirePackage{hyperref}

\PassOptionsToPackage{usenames,dvipsnames}{color} % color is loaded by hyperref
\hypersetup{unicode=true,
            colorlinks=true,
            linkcolor=Blue,
            citecolor=OliveGreen,
            urlcolor=Maroon,
            breaklinks=true,
            bookmarksdepth=3,
            hypertexnames=false
            }

\urlstyle{same}  % don't use monospace font for urls






%%%%%%%%%%%%%%%%%%%%%%%%%%%%%%%%%%%%%%%%%%%%%%%%%%%%%%%%%%%%%
% Commands for lists etc..

\setlength{\emergencystretch}{3em}  % prevent overfull lines
\providecommand{\tightlist}{%
  \setlength{\itemsep}{0pt}\setlength{\parskip}{0pt}}
\setcounter{secnumdepth}{0}
% Redefines (sub)paragraphs to behave more like sections
\ifx\paragraph\undefined\else
\let\oldparagraph\paragraph
\renewcommand{\paragraph}[1]{\oldparagraph{#1}\mbox{}}
\fi

\ifx\subparagraph\undefined \else
\let\oldsubparagraph\subparagraph
\renewcommand{\subparagraph}[1]{\oldsubparagraph{#1}\mbox{}}
\fi


\def\subsubsection{\subsection}







%%%%%%%%%%%%%%%%%%%%%%%%%%%%%%%%%%%%%%%%%%%%%%%%
% Macros

\newcommand*{\Z}{{\mathbb{Z}}}
\newcommand*{\R}{{\mathbb{R}}}
\newcommand*{\N}{{\mathbb{N}}}
\DeclareMathOperator*{\E}{\mathbb E}
\newcommand{\ceil}[1]{\lceil #1 \rceil}
\newcommand{\floor}[1]{\lfloor #1 \rfloor}

%% change \[ \] to numbered equation environment instead of equation*
\DeclareRobustCommand{\[}{\begin{equation}}
\DeclareRobustCommand{\]}{\end{equation}}






%\providecommand{\tightlist}{%
%  \setlength{\itemsep}{0pt}\setlength{\parskip}{0pt}}






%%%%%%%%%%%%%%%%%%%%%%%%%%%%%%%%%
% Title font format
% Setup all headings



\RequirePackage{titlesec}


\titleformat{\part}%
  [display]% shape
  {\begin{fullwidth}}% format applied to label+text
  {\normalfont\Huge\sffamily\bfseries\color{ocre}\thepart}% label
  {0pt}% horizontal separation between label and title body
  {\color{ocre}\Huge\sffamily \ucase{#1}}% before the title body
  [\end{fullwidth}]% after the title body



\titleformat{\chapter}%
  [display]% shape
  {\begin{fullwidth}}% format applied to label+text
  {\normalfont\itshape\huge\thechapter}% label
  {0pt}% horizontal separation between label and title body
  {\huge\rmfamily\itshape #1}% before the title body
  [\end{fullwidth}]% after the title body


\titleformat{\section}
  {\normalfont\large\sffamily\bfseries\color{ocre}}
  {\thesection}
  {0.5em}
  {\ucase{#1}}
  [\normalfont]
\titleformat{name=\section,numberless}
  {\normalfont\large\sffamily\bfseries\color{ocre}}
  {}
  {0em}
  {\ucase{#1}}
  [\normalfont]

\titleformat{\subsection}
  {\normalfont\sffamily\bfseries}
  {\thesubsection}
  {0.5em}
  {#1}
  [\normalfont]
\titleformat{\subsubsection}
  {\normalfont\sffamily\small\bfseries\itshape}
  {\thesubsubsection}
  {0.5em}
  {#1}
  [\normalfont]
\titleformat{\paragraph}[runin]
  {\normalfont\sffamily\small\bfseries}
  {}
  {0em}
  {#1}
  [\normalfont]


\titlespacing*{\section}{0pc}{3ex \@plus4pt \@minus3pt}{5pt}
\titlespacing*{\subsection}{0pc}{2.5ex \@plus3pt \@minus2pt}{0pt}
\titlespacing*{\subsubsection}{0pc}{2ex \@plus2.5pt \@minus1.5pt}{0pt}
\titlespacing*{\paragraph}{0pc}{1.5ex \@plus2pt \@minus1pt}{10pt}



%%%%%%%%%%%%%%%%%%%%%%%%%%%%%%%%%%
% Options for ToC
% tocloft package

\renewcommand\cftsecnumwidth{3em}
\renewcommand\cftsubsecnumwidth{3.5em}


%%%%%%%%%%%%%%%%%%%%%%%%%%%%%%%%%%%%





%%
% Prints argument within hanging parentheses (i.e., parentheses that take
% up no horizontal space).  Useful in tabular environments.
\newcommand{\hangp}[1]{\makebox[0pt][r]{(}#1\makebox[0pt][l]{)}}

%%
% Prints an asterisk that takes up no horizontal space.
% Useful in tabular environments.
\newcommand{\hangstar}{\makebox[0pt][l]{*}}



% Prints the month name (e.g., January) and the year (e.g., 2008)
\newcommand{\monthyear}{%
  \ifcase\month\or January\or February\or March\or April\or May\or June\or
  July\or August\or September\or October\or November\or
  December\fi\space\number\year
}


% Inserts a blank page
\newcommand{\blankpage}{\newpage\hbox{}\thispagestyle{empty}\newpage}




\newcommand{\compiletime}{\begin{textblock}{6}(0.5,15.5) {\footnotesize {\color{gray} Compiled on \the\month.\the\day.\the\year \ \currenttime}} \end{textblock}
}




\makeatletter
\let\stdchapter\chapter
\renewcommand*\chapter{%
  \@ifstar{\starchapter}{\@dblarg\nostarchapter}}
\newcommand*\starchapter[1]{\stdchapter*{#1} \compiletime}
\def\nostarchapter[#1]#2{\stdchapter[{#1}]{#2} \compiletime}
\makeatother



%%%%%%%%%%%%%%%%%%%%%%%%%%%%%%%%%%%%%%%%%%%%%%%%%%%%%%%
% Set fonts
%%%%%%%%%%%%%%%%%%%%%%%%%%%%%%%%%%%%%%%%%%%%%%%%%%%%%%%

\newif\ifunicode
\ifxetex\unicodetrue\fi
\ifluatex\unicodetrue\fi


% See https://github.com/Tufte-LaTeX/tufte-latex/issues/107

\ifunicode
  \newcommand{\textls}[2][5]{%
    \begingroup\addfontfeatures{LetterSpace=#1}#2\endgroup
  }
  \renewcommand{\allcapsspacing}[1]{\textls[15]{#1}}
  \renewcommand{\smallcapsspacing}[1]{\textls[10]{#1}}
  \renewcommand{\allcaps}[1]{\textls[15]{\ucase{#1}}}
  \renewcommand{\smallcaps}[1]{\smallcapsspacing{\scshape\MakeTextLowercase{#1}}}
  \renewcommand{\textsc}[1]{\smallcapsspacing{\textsmallcaps{#1}}}



  \usepackage{fontspec,unicode-math}
  \usepackage{xunicode}

 \DeclareTextCommandDefault{\nobreakspace}{\leavevmode\nobreak\ }
% from https://tex.stackexchange.com/questions/66949/command-nobreakspace-unavailable-when-switching-to-t1-encoding-under-xelatex/66951#66951



%\setmainfont[Ligatures=TeX]{BergamoPro-Regular.otf}[
%BoldFont = BergamoPro-Bold.otf ,
%ItalicFont = BergamoPro-Italic.otf ,
%BoldItalicFont = BergamoPro-BoldItalic.otf ]



%  \setmainfont[Ligatures=TeX]{Bergamo Pro} % hopefully available on system
  \setromanfont[Ligatures=TeX]{TeX Gyre Pagella}
  \setsansfont[Ligatures=TeX,Scale=MatchLowercase]{TeX Gyre Heros}

  %\setmonofont{FreeMono} % other option with many characters

  %\usepackage{inconsolata}
  %\setmonofont[Scale=MatchLowercase]{Inconsolata}

  \setmonofont[Scale=MatchLowercase]{Inconsolata LGC Markup}
  
  \usepackage{newunicodechar}

  \newfontfamily{\fallbackmono}{FreeMono}

  \newunicodechar{〈}{$\langle$}
  \newunicodechar{〉}{$\rangle$}
  %\newunicodechar{〈}{%
  %\texttt{\makebox[\fontcharwd\font`a]{◦}}%
  %}

  \setmathfont{Latin Modern Math}
  \setmathfont[range=\varnothing]{Asana Math}
  \setmathfont[range=\setminus]{Asana Math}
  %\setmathfont[range=\int]{Latin Modern Math}




\fi



\ifunicode
\else
\errmessage{At the moment there's only support for xelatex}
\end
\fi



%%%%%%%%%%%%%%%%%%%%%%%%%%%%%%%%%%%%%%%%%%%%%%%%%%%%%%%
% Load theorem environments



\usepackage{tikz}
\definecolor{shadecolor}{rgb}{0.9,0.9,0.9}
\definecolor{ocre}{RGB}{243,102,25}
\definecolor{examplecolor}{RGB}{218, 242, 193}
\definecolor{bigideacolor}{RGB}{255, 255, 86}
\definecolor{algorithmcolor}{RGB}{216, 231, 255}

%----------------------------------------------------------------------------------------
%	THEOREM STYLES
%----------------------------------------------------------------------------------------


\usepackage{amsmath,amsfonts,amssymb,amsthm} % For math equations, theorems, symbols, etc

\newcommand{\intoo}[2]{\mathopen{]}#1\,;#2\mathclose{[}}
\newcommand{\ud}{\mathop{\mathrm{{}d}}\mathopen{}}
\newcommand{\intff}[2]{\mathopen{[}#1\,;#2\mathclose{]}}
\newtheorem{notation}{Notation}[chapter]


%\newtheoremstyle{<name>}% name
%  {<dimen>}%              Space above
%  {<dimen>}%              Space below
%  {<font>}%               Body font
%  {}%                     Indent amount (empty = no indent, \parindent = para indent)
%  {<font>}%               Thm head font
%  {<punct>}%              Punctuation after thm head
%  {<dimen>}%              Space after thm head: " " = normal interword space; \newline = linebreak
%  {<spec>}%               Thm head spec (can be left empty, meaning `normal')

\makeatletter

% Boxed/framed environments
\newtheoremstyle{ocrenumbox}% % Theorem style name
{1em}% Space above was 0pt
{1ex}% Space below was 0pt
{\normalfont}% % Body font
{}% Indent amount
{\small\bf\sffamily\color{ocre}}% % Theorem head font
{\;}% Punctuation after theorem head
{0.25em}% Space after theorem head
{\small\sffamily\color{ocre}\thmname{#1}\nobreakspace\thmnumber{\@ifnotempty{#1}{}\@upn{#2}}% Theorem text (e.g. Theorem 2.1)
\thmnote{\nobreakspace\the\thm@notefont\sffamily\bfseries\color{black}---\nobreakspace#3.}} % Optional theorem note
\renewcommand{\qedsymbol}{$\blacksquare$}% Optional qed square


\newtheoremstyle{bigideabox}% % Theorem style name
{1em}% Space above was 0pt
{1ex}% Space below was 0pt
{\normalfont}% % Body font
{}% Indent amount
{\Large\bf\sffamily}% % Theorem head font
{\;}% Punctuation after theorem head
{\newline}% Space after theorem head
{\Large \sffamily \faLightbulbO   \hspace{0.3em} \thmname{#1}\nobreakspace\thmnumber{\@ifnotempty{#1}{}\@upn{#2}}% Theorem text (e.g. Theorem 2.1)
\thmnote{\nobreakspace\the\thm@notefont\sffamily\bfseries\color{black}---\nobreakspace#3.}
\newline} % Optional theorem note


\newtheoremstyle{algorithmstyle} % Theorem style name
{0pt}% Space above
{0pt}% Space below
{\normalfont}% Body font
{}% Indent amount
{\small\bf\sffamily}% Theorem head font
{ }% Punctuation after theorem head
{\newline}% Space after theorem head
{\small\sffamily\thmname{#1}\nobreakspace\thmnumber{\@ifnotempty{#1}{}\@upn{#2}} % Theorem text (e.g. Theorem 2.1)
\thmnote{\nobreakspace\the\thm@notefont\sffamily\bfseries---\nobreakspace#3.}}% Optional theorem



\newtheoremstyle{blacknumex}% Theorem style name
{5pt}% Space above
{5pt}% Space below
{\normalfont}% Body font
{} % Indent amount
{\small\bf\sffamily}% Theorem head font
{\;}% Punctuation after theorem head
{0.25em}% Space after theorem head
{\small\sffamily{\tiny\ensuremath{\blacksquare}}\nobreakspace\thmname{#1}\nobreakspace\thmnumber{\@ifnotempty{#1}{}\@upn{#2}}% Theorem text (e.g. Theorem 2.1)
\thmnote{\nobreakspace\the\thm@notefont\sffamily\bfseries---\nobreakspace#3.}}% Optional theorem note

\newtheoremstyle{blacknumbox} % Theorem style name
{0pt}% Space above
{0pt}% Space below
{\normalfont}% Body font
{}% Indent amount
{\small\bf\sffamily}% Theorem head font
{\;}% Punctuation after theorem head
{0.25em}% Space after theorem head
{\small\sffamily\thmname{#1}\nobreakspace\thmnumber{\@ifnotempty{#1}{}\@upn{#2}}% Theorem text (e.g. Theorem 2.1)
\thmnote{\nobreakspace\the\thm@notefont\sffamily\bfseries---\nobreakspace#3.}}% Optional theorem note

% Non-boxed/non-framed environments
\newtheoremstyle{ocrenum}% % Theorem style name
{1ex}% Space above was 5pt
{1ex}% Space below was 5pt
{\normalfont}% % Body font
{}% Indent amount
{\small\bf\sffamily\color{ocre}}% % Theorem head font
{\;}% Punctuation after theorem head
{0.25em}% Space after theorem head
{\small\sffamily\color{ocre}\thmname{#1}\nobreakspace\thmnumber{\@ifnotempty{#1}{}\@upn{#2}}% Theorem text (e.g. Theorem 2.1)
\thmnote{\nobreakspace\the\thm@notefont\sffamily\bfseries\color{black}---\nobreakspace#3.}} % Optional theorem note
\renewcommand{\qedsymbol}{$\blacksquare$}% Optional qed square


% Defines the theorem text style for each type of theorem to one of the three styles above
\newcounter{bigidea}

\newcounter{dummy}
\numberwithin{dummy}{chapter}

\theoremstyle{bigideabox}
\newtheorem{bigideaT}[bigidea]{Big Idea}

\theoremstyle{ocrenumbox}
\newtheorem{theoremeT}[dummy]{Theorem}
\newtheorem{problem}[dummy]{Problem}
\newtheorem{definitionT}[dummy]{Definition}
\newtheorem{remarkT}[dummy]{Remark}

\theoremstyle{algorithmstyle}
\newtheorem{algorithmT}[dummy]{Algorithm}

\theoremstyle{blacknumex}

\newtheorem{exampleT}[dummy]{Example}


\theoremstyle{blacknumbox}
\newtheorem{vocabulary}{Vocabulary}[chapter]
\newtheorem{corollaryT}[dummy]{Corollary}

\theoremstyle{ocrenum}
\newtheorem{exerciseT}{Exercise}[chapter]
\newtheorem{solvedexerciseT}{Solved Exercise}[chapter]
\newtheorem{proposition}[dummy]{Proposition}
\newtheorem{lemma}[dummy]{Lemma}  % added by Boaz


%----------------------------------------------------------------------------------------
%	DEFINITION OF COLORED BOXES
%----------------------------------------------------------------------------------------

\RequirePackage[framemethod=default]{mdframed} % Required for creating the theorem, definition, exercise and corollary boxes


%\usepackage{footnotehyper}
%\usepackage{footnote}

\def\temptexti{default value i}
\def\temptextii{default value ii}
\def\temptextiii{default value iii}

\newcounter{footnotetextcounter}

\newcommand{\tempfootnote}[1]{\stepcounter{footnotetextcounter}
 \footnotemark
 \global \expandafter\def\csname temptext\roman{footnotetextcounter}\endcsname{#1}
 }

\newenvironment{savenotes}
{\setcounter{footnotetextcounter}{0}
 \let\oldfootnote\footnote
 \let\footnote\tempfootnote
}
{\let\footnote\oldfootnote
 \spewnotes
 }

\newcommand{\spewnotes}{
\ifnum\value{footnotetextcounter}=1
\footnotetext{\temptexti}
\fi
\ifnum\value{footnotetextcounter}=2
\footnotetext{\temptexti}
\footnotetext{\temptextii}
\fi
\ifnum\value{footnotetextcounter}=3
\footnotetext{\temptexti}
\footnotetext{\temptextii}
\footnotetext{\temptextiii}
\fi
}

\newmdenv[
   leftmargin=\leftmargin,
   rightmargin=\leftmargin,
   topline=false,
   bottomline=false,
   skipabove=\topsep,
   skipbelow=\topsep
]{mdquote}

\iffalse

\let\origquote=\quote
\def\quote\mdquote
\def\endquote\endmdquote

\fi



% Theorem box
\newmdenv[
suppressfirstparskip=true,
skipabove=7pt,
skipbelow=7pt,
backgroundcolor=black!5,
linecolor=ocre,
innerleftmargin=5pt,
innerrightmargin=5pt,
innertopmargin=5pt,
leftmargin=0cm,
rightmargin=0cm,
innerbottommargin=5pt]{tBox}



% Big idea box
\newmdenv[
suppressfirstparskip=true,
skipabove=7pt,
skipbelow=7pt,
backgroundcolor=bigideacolor,
linecolor=black,
topline = true,
leftline = true,
rightline = true,
bottomline = true,
innerleftmargin=5pt,
innerrightmargin=5pt,
innertopmargin=5pt,
leftmargin=0cm,
rightmargin=0cm,
innerbottommargin=5pt]{biBox}



% Exercise box
\newmdenv[suppressfirstparskip=true,
skipabove=7pt,
skipbelow=7pt,
rightline=false,
leftline=true,
topline=false,
bottomline=false,
backgroundcolor=ocre!10,
linecolor=ocre,
innerleftmargin=5pt,
innerrightmargin=5pt,
innertopmargin=5pt,
innerbottommargin=5pt,
leftmargin=0cm,
rightmargin=0cm,
linewidth=4pt]{eBox}

% Example box
\newmdenv[suppressfirstparskip=true,
skipabove=7pt,
skipbelow=7pt,
rightline=false,
leftline=true,
topline=false,
bottomline=false,
backgroundcolor=examplecolor!10,
linecolor=ocre,
innerleftmargin=5pt,
innerrightmargin=5pt,
innertopmargin=5pt,
innerbottommargin=5pt,
leftmargin=0cm,
rightmargin=0cm,
linewidth=4pt]{exBox}


% Algorithm box
\newmdenv[suppressfirstparskip=false,
skipabove=7pt,
skipbelow=7pt,
rightline=false,
leftline=true,
topline=false,
bottomline=false,
backgroundcolor=algorithmcolor,
linecolor=black,
innerleftmargin=5pt,
innerrightmargin=5pt,
innertopmargin=5pt,
innerbottommargin=5pt,
leftmargin=0cm,
rightmargin=0cm,
linewidth=4pt,
nobreak=true]{algBox}


% Definition box
\newmdenv[suppressfirstparskip=true,
skipabove=7pt,
skipbelow=7pt,
rightline=false,
leftline=true,
topline=false,
bottomline=false,
linecolor=ocre,
innerleftmargin=5pt,
innerrightmargin=5pt,
innertopmargin=0pt,
leftmargin=0cm,
rightmargin=0cm,
linewidth=4pt,
innerbottommargin=0pt]{dBox}

% Corollary box
\newmdenv[suppressfirstparskip=true,
skipabove=7pt,
skipbelow=7pt,
rightline=false,
leftline=true,
topline=false,
bottomline=false,
linecolor=gray,
backgroundcolor=black!5,
innerleftmargin=5pt,
innerrightmargin=5pt,
innertopmargin=5pt,
leftmargin=0cm,
rightmargin=0cm,
linewidth=4pt,
innerbottommargin=5pt]{cBox}

% Creates an environment for each type of theorem and assigns it a theorem text style from the "Theorem Styles" section above and a colored box from above
\newenvironment{theorem}{\begin{savenotes}\begin{tBox}\begin{theoremeT}}{\end{theoremeT}\end{tBox}\end{savenotes}}


%\newenvironment{bigidea}{\begin{savenotes}\begin{biBox}\begin{bigideaT}}{\end{bigideaT}\end{biBox}\end{savenotes}}

%\newenvironment{bigidea}{\begin{savenotes}\begin{mybigideabox}\begin{bigideaT}}{\end{bigideaT}\end{mybigideabox}\end{savenotes}}


\newenvironment{bigidea}{\medskip \begin{shadowideablock}{\textwidth}\begin{bigideaT}}{\end{bigideaT}\end{shadowideablock} \medskip}




%\newenvironment{exercise}{\begin{savenotes}\begin{cBox}\begin{exerciseT}}{\hfill{\color{ocre}\tiny\ensuremath{\blacksquare}}\end{exerciseT}\end{cBox}\end{savenotes}}
\newenvironment{exercise}{\begin{exerciseT}}{\hfill{\color{ocre}\tiny\ensuremath{\blacksquare}}\end{exerciseT}}

\newenvironment{solvedexercise}{\begin{solvedexerciseT}}{\hfill{\color{ocre}\tiny\ensuremath{\blacksquare}}\end{solvedexerciseT}}

\newenvironment{solution}{\begin{dBox} \medskip \noindent {\small\bf\sffamily\color{ocre} Solution:}}{\hfill{\color{ocre}\tiny\ensuremath{\blacksquare}}\end{dBox}}



\newenvironment{definition}{\begin{savenotes}\begin{eBox}\begin{definitionT}}{\end{definitionT}\end{eBox}\end{savenotes}}
\newenvironment{example}{\begin{savenotes}\begin{exBox}\begin{exampleT}}{\end{exampleT}\end{exBox}\end{savenotes}}

\newenvironment{oldalgorithm}{\begin{savenotes}\begin{algBox}\begin{algorithmT}}{\end{algorithmT}\end{algBox}\end{savenotes}}

\newenvironment{algorithm}{\begin{savenotes}\begin{algBox}\begin{algorithmT}}{\end{algorithmT}\end{algBox}\end{savenotes}}


\newenvironment{corollary}{\begin{savenotes}\begin{cBox}\begin{corollaryT}}{\end{corollaryT}\end{cBox}\end{savenotes}}


%---------------------------
% SHADED QUOTE (added by Boaz)
%----------------------------
\definecolor{my-light-blue}{RGB}{192, 228 , 238}

\definecolor{my-light-yellow}{RGB}{247, 240, 195}

\definecolor{my-light-pink}{RGB}{239, 210, 239}

\definecolor{my-light-green}{RGB}{248, 255, 237}



\mdfdefinestyle{MyShadeQuoteStyle}{%
    leftmargin=15pt,
    rightmargin=15pt,
    backgroundcolor=my-light-green,
    linewidth=0pt,
    skipbelow=\topskip,
    skipabove=\topskip
}



\newenvironment{MyShadequote}[1][]{%
    \ignorespaces%
    \begin{mdframed}[style=MyShadeQuoteStyle,#1]%
}{%
    \end{mdframedfoot}%
    \ignorespacesafterend%
}%


\usepackage[most]{tcolorbox}
\definecolor{block-gray}{gray}{0.85}
\newtcolorbox{myshadedquotetwo}{colback=my-light-green,grow to right by=-10mm,grow to left by=-10mm, boxrule=0pt,boxsep=0pt,breakable}

% 192, 228 , 238
% 0.75, 0.89, 0.93

\newtcolorbox{mypausebox}{colback=my-light-blue,grow to right by=-3mm,grow to left by=-3mm, boxrule=0pt,boxsep=0pt,breakable}

\newtcolorbox{myremarkbox}{colback=my-light-yellow,grow to right by=-3mm,grow to left by=-3mm, boxrule=0pt,boxsep=0pt,breakable}


\newtcolorbox{myalgorithmbox}{colback=algorithmcolor,grow to right by=-3mm,grow to left by=-3mm, boxrule=0pt,boxsep=0pt,breakable}



\newtcolorbox{myrecapbox}{colback=my-light-pink,grow to right by=-3mm,grow to left by=-3mm, boxrule=0pt,boxsep=0pt,breakable}

\newtcolorbox{myobjectivesbox}{colback=my-light-pink,grow to right by=-3mm,grow to left by=-3mm, boxrule=0pt,boxsep=0pt,breakable}

\newtcolorbox{mybigideabox}{colback=bigideacolor,grow to right by=-3mm,grow to left by=-3mm, boxrule=0pt,boxsep=0pt,breakable}


\newenvironment{pausebox}{\begin{savenotes}\begin{mypausebox}}{\end{mypausebox}\end{savenotes}}
\newenvironment{remarkbox}{\begin{savenotes}\begin{myremarkbox}}{\end{myremarkbox}\end{savenotes}}
\newenvironment{recapbox}{\begin{savenotes}\begin{myrecapbox}}{\end{myrecapbox}\end{savenotes}}
\newenvironment{algorithmbox}{\begin{savenotes}\begin{myalgorithmbox}}{\end{myalgorithmbox}\end{savenotes}}

\newenvironment{objectivesbox}{\begin{savenotes}\begin{myobjectivesbox}}{\end{myobjectivesbox}\end{savenotes}}


%\renewenvironment{quote}%
%{\begin{snugshade}\begin{oldquote}}
%{\hfill\end{oldquote}\end{snugshade}}


\let\oldquote\quote
\let\oldendquote\endquote
%\def\quote{\begin{savenotes}\begin{mdframed}[style=MyShadeQuoteStyle] \begingroup \oldquote}
%\def\endquote{\hfill \oldendquote \endgroup \end{mdframed}\end{savenotes}}

\def\quote{\begin{savenotes}\begin{myshadedquotetwo}\begingroup \oldquote}
\def\endquote{\hfill \oldendquote \endgroup \end{myshadedquotetwo}\end{savenotes}}


%----------------------------------------------------------------------------------------
%	REMARK AND PAUSE ENVIRONMENTS
%----------------------------------------------------------------------------------------

\newenvironment{remark}[1][\unskip]{\begin{remarkbox}\par\vspace{10pt}\small % Vertical white space above the remark and smaller font size
\begin{list}{}{
\leftmargin=35pt % Indentation on the left
\rightmargin=25pt}\item\ignorespaces % Indentation on the right
\makebox[-2.5pt]{\begin{tikzpicture}[overlay]
\node[draw=ocre!60,line width=1pt,circle,fill=ocre!25,font=\sffamily\bfseries,inner sep=2pt,outer sep=0pt] at (-15pt,0pt){\textcolor{ocre}{R}};\end{tikzpicture}
} % Orange R in a circle
\advance\baselineskip -1pt
{\small\bf\sffamily\color{ocre}} \begin{remarkT}[#1]
}{\end{remarkT} \end{list}\vskip5pt \end{remarkbox}} % Tighter line spacing and white space after remark



\newenvironment{recap}{\begin{recapbox}\par\vspace{10pt}\small % Vertical white space above the remark and smaller font size
\begin{list}{}{
\leftmargin=35pt % Indentation on the left
\rightmargin=25pt}\item\ignorespaces % Indentation on the right
\makebox[-2.5pt]{\begin{tikzpicture}[overlay]
\node[draw=ocre!60,line width=1pt,circle,fill=ocre!25,font=\sffamily\bfseries,inner sep=2pt,outer sep=0pt] at (-15pt,0pt){\textcolor{blue}{$\checkmark$}};\end{tikzpicture}
} % Checkmark in a circle
\advance\baselineskip -1pt
{\small\bf\sffamily\color{ocre} Lecture Recap}
}{\end{list}\vskip5pt \end{recapbox}} % Tighter line spacing and white space after remark


\newenvironment{pause}[1][\unskip]{\begin{pausebox}\par\vspace{10pt}\small % Vertical white space above the remark and smaller font size
\begin{list}{}{
\leftmargin=35pt % Indentation on the left
\rightmargin=25pt}\item\ignorespaces % Indentation on the right
\makebox[-2.5pt]{\begin{tikzpicture}[overlay]
\node[draw=blue!60,line width=1pt,circle,fill=blue!25,font=\sffamily\bfseries,inner sep=2pt,outer sep=0pt] at (-15pt,0pt){\textcolor{white}{P}};\end{tikzpicture}
} % Orange R in a circle
\advance\baselineskip -1pt
{\small\bf\sffamily\color{ocre} #1}
}{\end{list}\vskip5pt \end{pausebox}} % Tighter line spacing and white space after remark



%-----------------------------------------
% Learning objectives box
% Code copied from 	A Course in Machine Learning by Hal Daumé III
% http://ciml.info/
%------------------------------------------

\usepackage[absolute,overlay]{textpos}

%
% Boxed environment with semi-transparent shadow.
%
\newlength{\boxw}
\newlength{\boxh}
\newlength{\shadowsize}
\newlength{\boxroundness}
\newlength{\tmpa}
\newsavebox{\shadowblockbox}

\setlength{\shadowsize}{6pt}
\setlength{\boxroundness}{3pt}

\newenvironment{mycomment}{\begin{comment}}{\end{comment}}

\newenvironment{shadowblock}[1]%
{\begin{lrbox}{\shadowblockbox}\begin{minipage}{#1}}%
{\end{minipage}\end{lrbox}%
\settowidth{\boxw}{\usebox{\shadowblockbox}}%
\settodepth{\tmpa}{\usebox{\shadowblockbox}}%
\settoheight{\boxh}{\usebox{\shadowblockbox}}%
\addtolength{\boxh}{\tmpa}%
\begin{tikzpicture}
    \addtolength{\boxw}{\boxroundness * 2}
    \addtolength{\boxh}{\boxroundness * 2}

    \foreach \x in {0,.05,...,1}
    {
        \setlength{\tmpa}{\shadowsize * \real{\x}}
        \fill[xshift=\shadowsize - 1pt,yshift=-\shadowsize + 1pt,
                black,opacity=.04,rounded corners=\boxroundness]
            (\tmpa, \tmpa) rectangle +(\boxw - \tmpa - \tmpa,
                \boxh - \tmpa - \tmpa);
    }

    \filldraw[fill=my-light-pink, draw=black!50, rounded corners=\boxroundness]
        (0, 0) rectangle (\boxw, \boxh);
    \draw node[xshift=\boxroundness,yshift=\boxroundness,
        inner sep=1pt,outer sep=1pt,anchor=south west]
             (0,0) {\usebox{\shadowblockbox}};
\end{tikzpicture}}


\newenvironment{shadowideablock}[1]%
{\begin{lrbox}{\shadowblockbox}\begin{minipage}{#1}}%
{\end{minipage}\end{lrbox}%
\settowidth{\boxw}{\usebox{\shadowblockbox}}%
\settodepth{\tmpa}{\usebox{\shadowblockbox}}%
\settoheight{\boxh}{\usebox{\shadowblockbox}}%
\addtolength{\boxh}{\tmpa}%
\begin{tikzpicture}
    \addtolength{\boxw}{\boxroundness * 2}
    \addtolength{\boxh}{\boxroundness * 2}

    \foreach \x in {0,.05,...,1}
    {
        \setlength{\tmpa}{\shadowsize * \real{\x}}
        \fill[xshift=\shadowsize - 1pt,yshift=-\shadowsize + 1pt,
                black,opacity=.04,rounded corners=\boxroundness]
            (\tmpa, \tmpa) rectangle +(\boxw - \tmpa - \tmpa,
                \boxh - \tmpa - \tmpa);
    }

    \filldraw[fill=bigideacolor, draw=black!50, rounded corners=\boxroundness]
        (0, 0) rectangle (\boxw, \boxh);
    \draw node[xshift=\boxroundness,yshift=\boxroundness,
        inner sep=1pt,outer sep=1pt,anchor=south west]
             (0,0) {\usebox{\shadowblockbox}};
\end{tikzpicture}}


\newsavebox{\halobjectivesbox}
\newlength{\objectivesheight}
\newenvironment{objectives}
  {\begin{lrbox}{\halobjectivesbox}\begin{minipage}{2.5in}\vspace{2pt}{\bf Learning Objectives:}\begin{footnotesize}}
  {\end{footnotesize}\end{minipage}\end{lrbox}\begin{textblock}{2.5}(10.7,0.7)\begin{shadowblock}{2.5in}\usebox{\halobjectivesbox}\end{shadowblock}\end{textblock}\settoheight{\objectivesheight}{\usebox{\halobjectivesbox}}}

% 10.2, 2.5


\makeatother

\newenvironment{proofidea}{\medskip \noindent {\small\bf\sffamily\color{ocre} Proof Idea:}}{{\color{ocre}\ensuremath{\star}}\medskip\noindent}



\usepackage[capitalise,nameinlink]{cleveref}
\crefname{lemma}{Lemma}{Lemmas}
\crefname{definition}{Definition}{Definitions}
\crefname{problem}{Problem}{Problems}
\crefname{exercise}{Exercise}{Exercises}
\crefname{example}{Example}{Examples}
\crefname{remark}{Remark}{Remarks}
\crefname{solvedexercise}{Solved Exercise}{Solved Exercises}
\crefname{bigidea}{Big Idea}{Big Ideas}
\crefname{algorithm}{Algorithm}{Algorithms}





\ifnum\usehypertarget=1
\else
\renewcommand{\hypertarget}[2]{#2} % Boaz: don't use hypertarget
\fi


%%%%%%%%%%%%%%%%%%%%%%%%%%%%%%%%%%%%%%%%%%%%%%%%%%%%%%%%
% Pseudocode

\usepackage{algpseudocode}
%\usepackage{algorithm}
\let\WHILE\While\let\ENDWHILE\EndWhile%
   \let\FOR\For\let\FORALL\ForAll\let\ENDFOR\EndFor%
   \let\LOOP\Loop\let\ENDLOOP\EndLoop%
   \let\REPEAT\Repeat\let\UNTIL\Until%
   \let\PROCEDURE\Procedure\let\ENDPROCEDURE\EndProcedure%
   \let\FUNCTION\Function\let\ENDFUNCTION\EndFunction%
   \let\IF\If\let\ELSIF\ElsIf\let\ELSE\Else\let\ENDIF\EndIf%
   \let\REQUIRE\Require\let\ENSURE\Ensure%
   \let\STATE\State\let\STATEx\Statex%
   \let\COMMENT\Comment
   \let\CALL\Call
   \let\oldReturn\Return
    \renewcommand{\Return}{\State\oldReturn}
   \let\RETURN\Return
   

   \algrenewcommand\algorithmicrequire{\textbf{Input:}}
   \algrenewcommand\algorithmicensure{\textbf{Output:}}
   \let\INPUT\REQUIRE
   \let\OUTPUT\ENSURE

\newcommand{\commentsymbol}{\#}% or \% or $\triangleright$
\algrenewcommand\algorithmiccomment[1]{\hfill \textit{\color{blue} \commentsymbol{} #1}}

   


